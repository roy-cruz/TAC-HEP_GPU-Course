\documentclass{article}
\usepackage{graphicx}
\usepackage[colorlinks=true, allcolors=blue]{hyperref}
\usepackage[letterpaper,top=2cm,bottom=2cm,left=3cm,right=3cm,marginparwidth=1.75cm]{geometry}
\usepackage{graphicx}
\usepackage{indentfirst}
\usepackage{tcolorbox}
\usepackage{xcolor}
\usepackage{amsmath}

\definecolor{codegreen}{rgb}{0,0.6,0}
\definecolor{codegray}{rgb}{0.5,0.5,0.5}
\definecolor{codepurple}{rgb}{0.58,0,0.82}
\definecolor{backcolour}{rgb}{0.95,0.95,0.92}

\usepackage{listings}
\lstset{language=C++}
\lstset{breaklines=true}
\lstset{
    literate={"}{{\texttt{"}}}1
             {'}{{\texttt{'}}}1
             {~}{{\texttt{~}}}1
             { }{{\ }}1
}
\lstdefinestyle{mystyle}{
    backgroundcolor=\color{backcolour},   
    commentstyle=\color{codegreen},
    keywordstyle=\color{magenta},
    numberstyle=\tiny\color{codegray},
    stringstyle=\color{codepurple},
    basicstyle=\ttfamily\footnotesize,
    breakatwhitespace=false,         
    breaklines=true,                 
    captionpos=t,                    
    keepspaces=true,                 
    numbers=left,                    
    numbersep=5pt,                  
    showspaces=false,                
    showstringspaces=false,
    showtabs=false,                  
    tabsize=2
}
\lstset{style=mystyle}

\lstdefinestyle{outputstyle}{
    backgroundcolor=\color{gray!10},
    basicstyle=\ttfamily\footnotesize,
    frame=single,
    breaklines=true,
    showstringspaces=false,
    numbers=none
}

\newcounter{exercise}
\newenvironment{exr}[1]{%
    \refstepcounter{exercise}
    \begin{tcolorbox}[colback=blue!5!white, colframe=blue!75!black, title=Exercise \theexercise]
    \textbf{Instructions:} #1
    \end{tcolorbox}
    \vspace{1em}
}{}

\patchcmd{\thebibliography}{\section*{\refname}}{}{}{}

\title{
    TAC-HEP GPU Programming Training Module: Assignment 2
}

\author{Roy F. Cruz}

\begin{document}

\date{October 12, 2025}
\maketitle

The GitHub repository hosting this code can be found through the following link: \href{https://github.com/roy-cruz/TAC-HEP_GPU-Course_Assignments/tree/master}{link}

\begin{exr}{
    \begin{itemize}
        \item Write a CUDA kernel that swaps the elements of two vectors.
        \item Start from file \texttt{swap\_vectors.cu}.
    \end{itemize}
    }
\end{exr}

...

\lstinputlisting[language=C++, caption={\ttfamily{add\_matrix.cu}}]{../../assignment2/add_matrix.cu}

% ------------------------------------------------------------------------


\begin{exr}{
    \begin{itemize}
      \item Write a CUDA kernel that adds two matrices of size $N\times M$.
      \item Express the indices of the matrices in 1 dimension:
        \begin{itemize}
          \item In a 1-d array the 2-d matrix will look like:
            \begin{align*}
            [A_{11},\, A_{12},\, A_{13},\, \dots,\, A_{1M}, A_{21},\, A_{22},\, \dots,\, A_{2M}, \dots, A_{N1},\, A_{N2},\, \dots,\, A_{NM}]
            \end{align*}
          \item For a matrix of size $N\times M$ each element can be expressed as \\
            \texttt{2D:[i,j] = 1D:[i * M + j]}
        \end{itemize}
      \item Use 2-dimensions to express threads and blocks: \texttt{threadIdx.x}, \texttt{blockIdx.x}, \texttt{threadIdx.y}, \texttt{blockIdx.y}.
      \item Start from file \texttt{add\_matrix.cu}.
    \end{itemize}   
}\end{exr}

...

\lstinputlisting[language=C++, caption={\ttfamily{add\_matrix.cu}}]{../../assignment2/add_matrix.cu}


\begin{out}
Some elements of the arrays before swapping
A = 0.840188, 0.783099, 0.911647, 0.335223, 
B = 0.394383, 0.798440, 0.197551, 0.768230, 
Some elements of the arrays after swapping
A = 0.394383, 0.798440, 0.197551, 0.768230, 
\end{out}

% ------------------------------------------------------------------------



\begin{exr}{
    \begin{itemize}
      \item Write a C++ function that multiplies two matrices of size $N\times N$.
      \item Write a CUDA kernel that multiplies two matrices of size $N\times N$.
      \item You should express this using a 1-dimensional vector:
        \begin{itemize}
          \item In a 1-d array the 2-d matrix will look like: 
          \begin{align*}
              [A_{11},\, A_{12},\, A_{13},\, \dots,\, A_{1N}, A_{21},\, A_{22},\, \dots,\, A_{2N}, \dots, A_{N1},\, A_{N2},\, \dots,\, A_{NN}]
          \end{align*}
        \end{itemize}
      \item In the kernel use 2-dimensions to express threads and blocks: \texttt{threadIdx.x}, \texttt{blockIdx.x} and \texttt{threadIdx.y}, \texttt{blockIdx.y}.
      \item Start from file \texttt{mult\_matrix.cu}.
      \item Add error checking:
        \begin{itemize}
          \item After allocating memory
          \item After copying from host to device
          \item After launching kernel
          \item After copying from device to host
        \end{itemize}
      \item Use a timer to measure how long the matrix multiplication takes on the CPU.
      \item Use a timer to measure how long the matrix multiplication takes on the GPU.
    \end{itemize}
}\end{exr}

...

\lstinputlisting[language=C++, caption={\texttt{mult\_matrix.cu}}]{../../assignment2/mult_matrix.cu}

\end{document}