\documentclass{article}
\usepackage{graphicx}
\usepackage[colorlinks=true, allcolors=blue]{hyperref}
\usepackage[letterpaper,top=2cm,bottom=2cm,left=3cm,right=3cm,marginparwidth=1.75cm]{geometry}
\usepackage{graphicx}
\usepackage{indentfirst}
\usepackage{tcolorbox}
\usepackage{xcolor}
\usepackage{amsmath}

\definecolor{codegreen}{rgb}{0,0.6,0}
\definecolor{codegray}{rgb}{0.5,0.5,0.5}
\definecolor{codepurple}{rgb}{0.58,0,0.82}
\definecolor{backcolour}{rgb}{0.95,0.95,0.92}

\usepackage{listings}
\lstset{language=C++}
\lstset{breaklines=true}
\lstset{
    literate={"}{{\texttt{"}}}1
             {'}{{\texttt{'}}}1
             {~}{{\texttt{~}}}1
             { }{{\ }}1
}
\lstdefinestyle{mystyle}{
    % backgroundcolor=\color{backcolour},   
    commentstyle=\color{codegreen},
    keywordstyle=\color{magenta},
    numberstyle=\tiny\color{codegray},
    stringstyle=\color{codepurple},
    basicstyle=\ttfamily\footnotesize,
    breakatwhitespace=false,         
    breaklines=true,                 
    captionpos=t,                    
    keepspaces=true,                 
    numbers=left,                    
    numbersep=5pt,                  
    showspaces=false,                
    showstringspaces=false,
    showtabs=false,                  
    tabsize=2
}
\lstset{style=mystyle}

\newcounter{exercise}
\newenvironment{exr}[1]{%
    \refstepcounter{exercise}
    \begin{tcolorbox}[colback=blue!5!white, colframe=blue!75!black, title=Exercise \theexercise]
    \textbf{Instructions:} #1
    \end{tcolorbox}
    \vspace{1em}
}{}

\patchcmd{\thebibliography}{\section*{\refname}}{}{}{}

\title{
    TAC-HEP GPU Programming Training Module: Assignment 1
}

\author{Roy F. Cruz}

\begin{document}

\date{October 5, 2025}
\maketitle

The GitHub repository hosting this code can be found through the following link: \href{https://github.com/roy-cruz/TAC-HEP_GPU-Course_Assignments/tree/master}{link}

\begin{exr}{
    \begin{itemize}
        \item Write a function that takes two integer arguments, a and b, and swaps their values.
        \item In the main function, create two arrays A and B, that hold 10 integer numbers each. Use the function that you wrote to swap all values of A to B and vice-versa.
    \end{itemize}
    }
\end{exr}

The function \texttt{swap}\_\texttt{vals} takes as input the address of two integer variables, \texttt{a} and \texttt{b}, and exchanges the value of two variables. It does this by first storing the value of \texttt{a} in a temporary variable \texttt{temp}, assigning the value of \texttt{b} to \texttt{a}, and then it assigns the value of the temporary variable to \texttt{b}. In order to swap the values of two arrays, this function is iteratively applied to each element of two arrays \texttt{A} and \texttt{B} of equal length.

\lstinputlisting[language=C++, caption={\ttfamily{exercise1.cc}}]{../../assignment1/exercise1.cc}

\begin{exr}{
    \begin{itemize}
        \item Create a data structure that will hold information about the TAC-HEP students such as name, email, username, experiment etc.
        \item From the main function create multiple objects, one for each student including yourselves.
        \item Create a function that takes the struct as an argument and prints out the values of all struct members. Make sure to pass the struct by reference and ensure that members values cannot change.
    \end{itemize}
    }
\end{exr}

For this exercise, a simple \texttt{struct} named \texttt{Student} was defined with elements of type \texttt{std::string} which was used to hold student information. A function named \texttt{print\_student} was also implemented that takes as input a (constant) \texttt{Student} and prints out each of the members of this \texttt{struct}. These implementations are used in the main function of the code by first instantiating some example \texttt{Student} variables, and then using the \texttt{print\_student} function to print the information they were initialized with.

\lstinputlisting[language=C++, caption={\ttfamily{exercise2.cc}}]{../../assignment1/exercise2.cc}

\begin{exr}{
    \begin{itemize}
        \item Create a function that simulates the game "rock, paper, scissors". The function should take as input the choice of both players (rock, paper or scissors). The function should return the player who won or if there was a draw.
    \end{itemize}
    }
\end{exr}

For this exercise, a simple function called \texttt{input\_rps} (i.e. "input rock, paper, scissor") was implemented, which is used twice to prompt the user to input the choice of each player. This information is encoded as integers, with scissors represented by 0, paper by 1, and rock by 2. Another function, \texttt{ play\_rps}, is then called, which takes each player choice as input and, depending on the values between the encoded choices, outputs 0 if the result is a draw, 1 if player 1 won or 2 if player 2 won. 

The determination of the winner is based on modular arithmetic. If the choices are the same then its considered a draw. On the other hand, player 1 is considered the winner if
\begin{equation*}
    (\text{player 1 choice} + 1) \bmod(3) = \text{player 2 choice} .
\end{equation*}
Otherwise, player 2 wins. The result of the game is then printed.

\lstinputlisting[language=C++, caption={\ttfamily{exercise3.cc}}]{../../assignment1/exercise3.cc}

\begin{exr}{
    \begin{itemize}
        \item Take a look at the `main.cc` code in the Exercise 4 directory. This code reads branches from a ROOT tree containing lepton and jet kinematic information. To see the type of each variable take a look at the headerfile `hh/t1.h`.
        \item For the `Particle` class, add 1 more constructor that takes 4 arguments (4-momentum).
        \item Fix all member functions adding the appropriate implementation (denoted with FIXME).
        \item Using the `Particle` class, create two daughter classes, one for leptons (e.g. Lepton) and one for jets (e.g. Jet).
        \begin{itemize}
            \item To the lepton daughter class, add an additional member function that reads / stores the lepton charge.
            \item To the jet daughter class, add an additional member function that reads / stores the jet hadron flavor ( FYI, the value of the hadron flavor is 5 if jets are originating from b-hadrons, 4 if jet is originating from c-hadron and 0 if jet is originating from u,d,s hadrons or gluons).
        \end{itemize}
        \item In the main function, access event jets and leptons (NOTE: these are arrays, so you should create an additional loop to sead the elements of each variable).
        \begin{itemize}
            \item For each jet, create a jet object and print out the values of all its members.
            \item For each lepton, create a lepton object and print out the values of all its members.
        \end{itemize}
    \end{itemize}
    }
\end{exr}

For the \texttt{Particle} class, the implemented constructor takes as input the elements of a 4-vector, i.e. the $\vec p$ and $E$ of the input particle. These values are then stored in the \texttt{p} array and also used to compute and store the energy, mass, $p_T$, $\phi$ and $\eta$ values, the last 4 of which are defined and implemented as:

\begin{align*}
    m &= \sqrt{E^2 - (p_x^2 + p_y^2 + p_z^2)}\\
    p_T &= \sqrt{p_x^2 + p_y^2}\\
    \phi &=  \operatorname{atan2}(p_y, p_x)\\
    \eta &= \frac12 \ln\left(\frac{E + pz}{E - pz}\right)
\end{align*}

Additionally, two methods, \texttt{Particle::sintheta} and \texttt{Particle::p4}, were implemented. The former, as the name suggests, computes

\begin{align*}
    \sin\theta = \frac{p_T}{|\vec p|} = \frac{p_T}{\sqrt{p_x^2 + p_y^2 + p_z^2}}
\end{align*}

\noindent where $\theta$ is the angle between the beamline and the direction of the momentum 3-vector. This computed value is then returned. The second method calculates and stores the components of the 4-vector from the $E$, $p_T$, $\eta$, and $\phi$. The computation for the last three of these is done through the following relationships:

\begin{align*}
    p_x &= p_T \cos\phi\\
    p_y &= p_T \sin\phi\\
    p_z &= p_T \sinh\eta
\end{align*}

\noindent Finally, for the \texttt{Particle} base class, the \texttt{Particle::print} method was modified in order to print all of the members of the base class, as instructed in the exercise.

Along with the implementation of the required methods for the base class \texttt{Particle}, two sub-classes, \texttt{Jet} and \texttt{Lepton}, were implemented which inherit all of the aforementioned methods and members, but also introduce a new member: \texttt{Q} for \texttt{Lepton} and \texttt{hadronFlav} for \texttt{Jet}. Associated setter methods were included for these members. Furthermore, given that the provided ROOT file for this exercise contain the $\eta$, $\phi$ and $p_T$ of these objects rather than the $\vec p$, a static method \texttt{FromPtEtaPhiE} was implemented in both sub-classes which serves as an alternate constructor method which calls the default \texttt{Particle} constructor, before using the previously implemented \texttt{Particle::p4} method along with the setter method for the corresponding new member of the sub-class (\texttt{Q} or \texttt{hadronFlav}), before returning the fully constructed object.

In the main body of the code, a loop iterates over each event stored in the provided ROOT file. For each event, we iteratively read each lepton (hadron), storing their kinematic variables and charge (hadron flavor), before printing all of the stored and computed member variables. Uninitialized leptons (jets) in the ROOT file are skipped, and the iterations over the event continued in order to keep the output tidy.

\lstinputlisting[language=C++, caption={\ttfamily{main.cc}}]{../../assignment1/Exercise4/main.cc}

\end{document}