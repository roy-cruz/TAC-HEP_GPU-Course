\documentclass{article}
\usepackage{graphicx}
\usepackage[colorlinks=true, allcolors=blue]{hyperref}
\usepackage[letterpaper,top=2cm,bottom=2cm,left=3cm,right=3cm,marginparwidth=1.75cm]{geometry}
\usepackage{graphicx}
\usepackage{indentfirst}
\usepackage{tcolorbox}
\usepackage{xcolor}
\usepackage{amsmath}

\definecolor{codegreen}{rgb}{0,0.6,0}
\definecolor{codegray}{rgb}{0.5,0.5,0.5}
\definecolor{codepurple}{rgb}{0.58,0,0.82}
\definecolor{backcolour}{rgb}{0.95,0.95,0.92}

\usepackage{listings}
\lstset{language=C++}
\lstset{breaklines=true}
\lstset{
    literate={"}{{\texttt{"}}}1
             {'}{{\texttt{'}}}1
             {~}{{\texttt{~}}}1
             { }{{\ }}1
}
\lstdefinestyle{code}{
    backgroundcolor=\color{backcolour},   
    commentstyle=\color{codegreen},
    keywordstyle=\color{magenta},
    numberstyle=\tiny\color{codegray},
    stringstyle=\color{codepurple},
    basicstyle=\ttfamily\footnotesize,
    breakatwhitespace=false,         
    breaklines=true,                 
    captionpos=t,                    
    keepspaces=true,                 
    numbers=left,                    
    numbersep=5pt,                  
    showspaces=false,                
    showstringspaces=false,
    showtabs=false,                  
    tabsize=2,
    language=C++
}

\lstdefinestyle{output}{
    backgroundcolor=\color{black!5},
    basicstyle=\ttfamily\small,
    breaklines=true,
    showstringspaces=false,
    frame=single,
    rulecolor=\color{black!20},
    numbers=none,
    captionpos=t,
    language=bash
}

\renewcommand{\lstlistingname}{Code block}


\newcounter{exercise}
\newenvironment{exr}[1]{%
    \refstepcounter{exercise}
    \begin{tcolorbox}[colback=blue!5!white, colframe=blue!75!black, title=Exercise \theexercise]
    \textbf{Instructions:} #1
    \end{tcolorbox}
    \vspace{1em}
}{}

\patchcmd{\thebibliography}{\section*{\refname}}{}{}{}

\title{
    TAC-HEP GPU Programming Training Module: Final project
}

\author{Roy F. Cruz}

\begin{document}

\date{December 11, 2025}
\maketitle

The GitHub repository that hosts this code can be accessed via the following link: \href{https://github.com/roy-cruz/TAC-HEP_GPU-Course_Assignments/tree/master}{link}


\section{C++ and CPU Profiling}
\begin{exr}{
    \begin{itemize}
        \item Start by writing a code in C++ that:
        \begin{itemize}
            \item Creates two 2-dimensional square matrices \texttt{A} and \texttt{B} of size \texttt{DSIZE >= 512}   and fills them with arbitrary integer values.
            \item Performs a 2-dimensional stencil operation on each matrix. You can use any radius size, but keep it \texttt{> 2}.
            \item Performs a matrix multiplication of the matrices after the stencil application.
            \item Make sure that you also add utility functions to check your results.
        \end{itemize}
        \item Profile your C++ code using the VTune profiler and identify the compute intensive parts.
    \end{itemize}
    }
\end{exr}

\lstinputlisting[caption={\texttt{stencil\_mult.cpp}}, style=code]{../../project/ex1-cpu/stencil_mult.cpp}

\begin{lstlisting}[style=output]
[rcruzcan@g37n01 project]$ vtune -report summary -result-dir ./profiling/cpu_hotspots/
vtune: Using result path `/mnt/ceph/home/rcruzcan/private/courses/GPUProg/tac-hep-gpus-roycruz/project/profiling/cpu_hotspots'
vtune: Executing actions 75 % Generating a report                              Elapsed Time: 5.248s
    CPU Time: 5.240s
        Effective Time: 5.240s
        Spin Time: 0s
        Overhead Time: 0s
    Total Thread Count: 1
    Paused Time: 0s

Top Hotspots
Function         Module            CPU Time  % of CPU Time(%)
---------------  ----------------  --------  ----------------
matrix_mult      stencil_mult_cpu    5.220s             99.6%
compute_stencil  stencil_mult_cpu    0.020s              0.4%
Collection and Platform Info
    Application Command Line: objs/stencil_mult_cpu
    Operating System: 6.1.156-1.el9.elrepo.x86_64 \S Kernel \r on an \m
    Computer Name: g37n01.hep.wisc.edu
    Result Size: 3.7 MB
    Collection start time: 06:39:47 11/12/2025 UTC
    Collection stop time: 06:39:53 11/12/2025 UTC
    Collector Type: User-mode sampling and tracing
    CPU
        Name: Unknown
        Frequency: 3.000 GHz
        Logical CPU Count: 64
        Cache Allocation Technology
            Level 2 capability: not detected
            Level 3 capability: available
\end{lstlisting}


% ------------------------------------------------------------------------

\section{Porting to CUDA}
\begin{exr}{
    \begin{itemize}
        \item Write the same application in CUDA:
        \begin{itemize}
            \item You should write a CUDA kernel that performs the stencil operation and one for the matrix multiplication.
            \item Initially make use of explicit memory copies from host to device and vice-versa and make use only of the default CUDA stream.
            \item Make sure to add utility functions for error checking and for verifying your results.
        \end{itemize}
        \item Profile your code using nsys and document/comment on the time spent in each CUDA API call. Also, make note of the time spent on host and device.
        \item Try switching from explicit memory copies to managed memory.
        \begin{itemize}
            \item Profile again using either nsys on ncu and comment on the performance of your application
        \end{itemize}
    \end{itemize}
}\end{exr}

\lstinputlisting[caption={\texttt{compute\_funcs\_ex2.h}}, style=code]{../../project/hh/compute_funcs_ex2.h}

\lstinputlisting[caption={\texttt{stencil\_mult\_explicit.cu}}, style=code]{../../project/ex2-cuda/stencil_mult_explicit.cu}
\lstinputlisting[caption={\texttt{stencil\_mult\_managed.cu}}, style=code]{../../project/ex2-cuda/stencil_mult_managed.cu}

% ------------------------------------------------------------------------
\section{Optimizing Performance in CUDA}
\begin{exr}{
    \begin{itemize}
        \item Optimize the performance of your code making use of non-default CUDA streams and shared memory.
        \item Once you have decided on the best approach, profile your application and compare the time spent in each API call and the overall timing of your application with your initial CUDA implementation.
    \end{itemize}
}\end{exr}

\lstinputlisting[caption={\texttt{compute\_funcs\_ex3.h}}, style=code]{../../project/hh/compute_funcs_ex3.h}

\lstinputlisting[caption={\texttt{stencil\_mult\_managed.cu}}, style=code]{../../project/ex3-cudaopt/stencil_mult_opt.cu}

% ------------------------------------------------------------------------
\section{Making Use of Alpaka}
\begin{exr}{
    \begin{itemize}
        \item Re-write your application making use of the Alpaka portability library.
        \item Describe the steps you had to follow to re-write your code.
    \end{itemize}
}\end{exr}


% ------------------------------------------------------------------------
\newpage
\appendix

\section{Setup Description}


\end{document}